\documentclass[12pt]{article}

\usepackage{amsmath, amsfonts, amssymb}

\usepackage{physics}

% Title Page
\title{Quantum Computing:\\ Report on Quantum Hardware}
\author{Scott Schoeller}
\date{}


\begin{document}
\pagenumbering{gobble}
\maketitle
\newpage

\begin{flushleft}
\section{Abstraction of a Quantum Computer}
The quantum hardware chapter of the National Academies Report, \textit{Quantum Computing: Progress and Prospects} \cite{grumbling_quantum_2019} first discusses a big-picture view of quantum hardware.

\subsection{Quantum Data Plane}
This part of the quantum computer is defined as the actual qubits, circuitry and other hardware built to contain the qubits. Systematic errors occur between qubits and are slow to change. Thus, system calibration is possible.

\subsection{Control \& Host Processors}
The control processor plane executes the appropriate sequence of quantum gates. This converts compiled code to running commands.

\section{Types of Qubits}
Trapped ion qubits were the first type of gate-base qubits. first being demonstrated in 1995. Laser or microwave sources are commonly used in this type of quantum computer.\\ 
Superconducting qubits are supercooled to very low temperatures to achieve the desired effects required for quantum computing.\\

\section{Quantum Hardware as of 2018}
\par{
Gate-base machines have an error rate below below 0.1\% for gates involving single qubits. For the situation involving many qubits, the situation is more complicated.
}
\par{
Technical challenges exist related to both manufacturing and operation of large quantum computer. Qubit quality, raw materials for the machines and controlling the computation process are paramount.
}

\newpage

\bibliography{hwreport.bib}
\bibliographystyle{acm}
\end{flushleft}
\end{document}          
